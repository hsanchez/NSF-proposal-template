% !TEX root = ../main.tex

\section*{Introduction} %
\label{sec:introduction}

%About a page or so describing the overall idea and motivating hypotheses.
%Provide a nice introduction to our problem and what we're going to be focusing on.
% The Project Description should provide a clear statement of the work to be undertaken and must include the objectives for the period of the proposed work and expected significance; the relationship of this work to the present state of knowledge in the field, as well as to work in progress by the PI under other support.
%
% The Project Description should outline the general plan of work, including the broad design of activities to be undertaken, and, where appropriate, provide a clear description of experimental methods and procedures. Proposers should address what they want to do, why they want to do it, how they plan to do it, how they will know if they succeed, and what benefits could accrue if the project is successful.  The project activities may be based on previously established and/or innovative methods and approaches, but in either case must be well justified.  These issues apply to both the technical aspects of the proposal and the way in which the project may make broader contributions.

% Introduce the system and mention that it's meant to AUGMENT human cognition
% Our vision for \pname\ is a hyper collaborative platform that manages the burdensome components of software development in order to enable human cognitive work to be exclusively utilized for creative and interesting problems
Our goal is to push the boundaries of what humans and machines can do with small amounts of synchronization.
Our vision is \pname, a platform that manages the burdensome components of software development in order to enable the future of human knowledge work to be exclusively utilized for creative and interesting problems.
Our approach unites research theories from distributed systems, psychology, and applied software engineering to develop a human computation model of disorderly distributed work.
% More goes here