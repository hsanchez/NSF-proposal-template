% When evaluating NSF proposals, reviewers will be asked to consider what the proposers want to do, why they want to do it, how they plan to do it, how they will know if they succeed, and what benefits could accrue if the project is successful. These issues apply both to the technical aspects of the proposal and the way in which the project may make broader contributions. To that end, reviewers will be asked to evaluate all proposals against two criteria:
%
% Intellectual Merit: The Intellectual Merit criterion encompasses the potential to advance knowledge; and
% Broader Impacts: The Broader Impacts criterion encompasses the potential to benefit society and contribute to the achievement of specific, desired societal outcomes.
% The following elements should be considered in the review for both criteria:
%
% What is the potential for the proposed activity to
% Advance knowledge and understanding within its own field or across different fields (Intellectual Merit); and
% Benefit society or advance desired societal outcomes (Broader Impacts)?
% To what extent do the proposed activities suggest and explore creative, original, or potentially transformative concepts?
% Is the plan for carrying out the proposed activities well-reasoned, well-organized, and based on a sound rationale? Does the plan incorporate a mechanism to assess success?
% How well qualified is the individual, team, or organization to conduct the proposed activities?
% Are there adequate resources available to the PI (either at the home organization or through collaborations) to carry out the proposed activities?
% Broader impacts may be accomplished through the research itself, through the activities that are directly related to specific research projects, or through activities that are supported by, but are complementary to, the project. NSF values the advancement of scientific knowledge and activities that contribute to achievement of societally relevant outcomes. Such outcomes include, but are not limited to: full participation of women, persons with disabilities, and underrepresented minorities in science, technology, engineering, and mathematics (STEM); improved STEM education and educator development at any level; increased public scientific literacy and public engagement with science and technology; improved well-being of individuals in society; development of a diverse, globally competitive STEM workforce; increased partnerships between academia, industry, and others; improved national security; increased economic competitiveness of the United States; and enhanced infrastructure for research and education.
% Additional Solicitation Specific Review Criteria
% Reviewers will be asked to:
% Comment on the extent to which the project scope identifies compelling and innovative research problem(s) responsive to FW-HTF, including the potential for contribution toward (a) transforming the frontiers of science and technology for human performance augmentation and workplace skill acquisition; (b) improving both worker quality of life and employer financial metrics; (c) enhancing the economic and social well-being of the country; and (d) addressing societal needs through research on learning and instruction in the context of augmentation.
% Comment on the collaborative and interdisciplinary nature of the proposed research and the Collaboration Plan proposed.

% Template for Formatting
% Template info for NSF Proposal
\documentclass[11pt]{article}

\usepackage[letterpaper,left=1in,right=1in,top=1in,bottom=1in]{geometry}
\usepackage{subfigure}
\usepackage{acronym}
\usepackage{algorithm}
\usepackage{algpseudocode}
\usepackage{amsmath}
\usepackage{amsfonts}
\usepackage{amssymb}
\usepackage{array}
\usepackage{cases}
\usepackage{comment}
\usepackage{epsfig}

\usepackage{amsthm}
\theoremstyle{definition}
\newtheorem{exmp}{Example}[section]

\usepackage{fancyhdr} % put information into footer
\fancypagestyle{plain}{%
 \fancyhf{} % clear all header and footer fields
% \fancyfoot[C]{\newline Use or disclosure of data contained on this sheet is subject to the restrictions on the cover page of this proposal}
 \renewcommand{\headrulewidth}{0.0pt}
 \renewcommand{\footrulewidth}{0.4pt}}
\pagestyle{plain}

\usepackage{titling}
\usepackage[inline]{enumitem}

\usepackage{color}
\usepackage{soul}
\usepackage[colorinlistoftodos,prependcaption,textsize=tiny]{todonotes}
\definecolor{lred}{RGB}{255, 153, 153}
\definecolor{lpurp}{RGB}{204, 204, 255}
\definecolor{lorange}{RGB}{255, 204, 102}
\definecolor{dorange}{RGB}{255, 204, 153}

\newcommand{\unsure}[2]{\todo[linecolor=red,backgroundcolor=red!25,bordercolor=red]{#1}\sethlcolor{lred} \texthl{#2}}
\newcommand{\change}[2]{\todo[linecolor=lpurp,backgroundcolor=lpurp,bordercolor=lpurp]{#1}\sethlcolor{lpurp} \texthl{#2}}
\newcommand{\info}[2]{\todo[linecolor=dorange,backgroundcolor=dorange,bordercolor=dorange]{#1}\sethlcolor{dorange}\texthl{#2}}
\newcommand{\improvement}[2]{\todo[linecolor=blue!20!white,backgroundcolor=cyan,bordercolor=blue!20!white]{#1}\sethlcolor{cyan} \texthl{#2}}
\newcommand{\thiswillnotshow}[2]{\todo[disable]{#1}#2}
\newcommand{\explainindetail}[2]{\todo[color=green!40]{#1}\sethlcolor{green}\texthl{#2}}

%% Contributors
\newcommand{\Constantine}{{\bfseries JohnC: }}

%% date
\usepackage{datetime2}

\usepackage{graphicx}
\usepackage{multirow}
\usepackage{pdfpages}
\usepackage{url}
\usepackage{wrapfig}
\usepackage{setspace}
 \let\oldthebibliography=\thebibliography
 \let\endoldthebibliography=\endthebibliography
  \renewenvironment{thebibliography}[1]{
   \begin{oldthebibliography}{#1}
      \setlength{\parskip}{0ex}
      \setlength{\itemsep}{0ex}
  }
  {
    \end{oldthebibliography}
  }
\bibliographystyle{plain}

% - - - - - Specialized Markup  - - - - -
% Omit This: Comment out large sections of text
% Usage: \omitthis{commented text}
\long\def\omitthis#1{\relax}
%
% - - ToDo: Clear markup in paper margin
% % Usage: \todo{``your comment''}
% \newcommand{\todo}[1]{ {\mbox{}\marginpar[\hfill$\bigodot$]{$\bigodot$\hfill} {\sc#1}}}

% Shorthand and Acronyms
\newcommand{\titleFull}{Disorderly Distributed Human Computation}
\newcommand{\titleShort}{Disorderly Distributed Humans}
\newcommand{\system}{Anti-entropy-Way}
\acrodef{SRI}{SRI International}
\newcommand{\pname}{\textit{Twiza}}

% Document Start and Title in Footer
\begin{document}
\lfoot{\titleFull}

% (a) Cover Sheet
% Mostly populated by NSF Fastlane - Administrative

% (b) Project Summary
% \rfoot{B-\thepage}
\rfoot{B-Project Summary}
%!TEX root = ../main.tex

\section*{Project Summary} %
\label{sec:projectSummary}

% No more than a page! Written in third person.
\noindent
\textbf{Overview.}
Hyper collaboration, or innovation-focused collaboration among different population groups, has emerged as a dominant trend among diverse global sectors, ranging from technology, to infrastructure, to education, to research, and to policy making.
% More goes here
The research is motivated by two primary questions:
\begin{enumerate*}[label=(\arabic*),itemsep=5pt]
  % \item What is the optimal group structure and composition in hyper
  % collaborative networks?
  \item One
\end{enumerate*} To address these questions, it is hypothesized that
\begin{enumerate*}[label=(\arabic*),itemsep=5pt]
  \item This method is the best...
\end{enumerate*}

\medskip\noindent
\textbf{Intellectual Merit.}
%% Potential of the proposed activity to advance knowledge.
This project is addressing WHAT? This research will contribute to three major literatures:
\begin{enumerate*}[label=(\arabic*),itemsep=5pt]
  \item One
  \item Two
  \item Three
\end{enumerate*}


\medskip\noindent
\textbf{Broader Impacts.}
Beyond scientific impact, there is a focus on transition that leverages collaborations with Google and GitHub; key industry partners for realizing a new platform to amplify open source development..
% More goes here

\medskip\noindent
\textbf{Keywords.} Disorderly distributed computation, human computation, hyper
collaboration, open source software development, skill modeling, exploit-explore.
\newpage

% (c) Table of Contents
% Auto-Generated by NSF

% ========== Solicitation-Specific Information ==========
% Future of Work Focus: New convergent research...
% to understand and develop the human-technology partnership,
% to design new technologies to augment human performance,
% to illuminate the emerging socio-technological landscape, and
% to foster lifelong and pervasive learning with technology.
%
% Projects should be framed in terms of the potential contribution toward
% (a) transforming the frontiers of science and technology for human performance augmentation and workplace skill acquisition;
% (b) improving both worker quality of life and employer financial metrics;
% (c) enhancing the economic and social well-being of the country; and
% (d) addressing societal needs through research on learning and instruction in the context of augmentation
% Travel: One PI from each institute per PI meeting (yearly)
% ========================================================

% (d) Description (15 pages)
% Fully describe the relevance of the project to the goals of NSF's FW-HTF Big Idea. In this regard, PIs must frame the proposed research in terms of the potential contribution toward (a) transforming the frontiers of science and technology for human performance augmentation and workplace skill acquisition; (b) improving both worker quality of life and employer financial metrics; (c) enhancing the economic and social well-being of the country; and (d) addressing societal needs through research on learning and instruction in the context of augmentation.
% Outline specific research questions, hypotheses, and gaps in science, engineering, and/or education knowledge responsive to FW-HTF Theme 1 or 2.
% Describe the vision of success for the proposal — specifically defining the project goals and the definition of a successful outcome.
\rfoot{D-Project Description}
% \setcounter{page}{1}
% D 1-2 (about 2 pages)
% - Explain Scenario (.5 page)
% - Explain benefits over SOA for scenario
% - Explain components that have to come together to realize the scenario (.5 page)
% - Explain the scientific goals required to meet the scenario
% - Explain abbreviated research agenda/questions to meet scientific goals
% !TEX root = ../main.tex

\section*{Introduction} %
\label{sec:introduction}

%About a page or so describing the overall idea and motivating hypotheses.
%Provide a nice introduction to our problem and what we're going to be focusing on.
% The Project Description should provide a clear statement of the work to be undertaken and must include the objectives for the period of the proposed work and expected significance; the relationship of this work to the present state of knowledge in the field, as well as to work in progress by the PI under other support.
%
% The Project Description should outline the general plan of work, including the broad design of activities to be undertaken, and, where appropriate, provide a clear description of experimental methods and procedures. Proposers should address what they want to do, why they want to do it, how they plan to do it, how they will know if they succeed, and what benefits could accrue if the project is successful.  The project activities may be based on previously established and/or innovative methods and approaches, but in either case must be well justified.  These issues apply to both the technical aspects of the proposal and the way in which the project may make broader contributions.

% Introduce the system and mention that it's meant to AUGMENT human cognition
% Our vision for \pname\ is a hyper collaborative platform that manages the burdensome components of software development in order to enable human cognitive work to be exclusively utilized for creative and interesting problems
Our goal is to push the boundaries of what humans and machines can do with small amounts of synchronization.
Our vision is \pname, a platform that manages the burdensome components of software development in order to enable the future of human knowledge work to be exclusively utilized for creative and interesting problems.
Our approach unites research theories from distributed systems, psychology, and applied software engineering to develop a human computation model of disorderly distributed work.
% More goes here
% \input{sections/intellectualMerit}
% D 3-4
% - Background on current crowd-sourcing and issues
%   - Work distribution and management
%   - Developer onboarding
% - Background on self-determination theory and motivation
%   - How these apply to distributed work
% !TEX root = ../main.tex

\section*{Related Work}
\label{sec:relatedwork}

Applying the principles of disorderly distribution to the orchestration of human-and-machine work is uncharted.
% More goes here
% D 5-11 Research Plan - Challenges and Approaches
% !TEX root = ../main.tex

\section*{Research Plan} %
\label{sec:research-plan}

% This is the bulk of the proposal. Background, Approach, Experiments and Predicted Bits
This project explores the development of a disorderly distributed model of human computation and its application to the domain of open source software development.
% More goes here
% D 12: Preliminary/Previous Results
% \input{sections/preliminaryResults}
% D 13: Team
%!TEX root = ../main.tex

\section*{Team Qualifications and Responsibilities} %
\label{sec:teamQuals}

Our team is composed of researchers from broad disciplines across a non-profit research
institute and a liberal arts college. The SRI team is providing primary theoretical
motivation from distributed systems and psychology, and will also support the majority of
the technical requirements for project management. Pomona College will be the primary
driver of experimentation and evaluation of the system and research hypotheses, and will
provide mentoring for students a postdoctoral researcher.
% More goes here

% D 14-15: Broader Impacts
% !TEX root = ../main.tex

\section*{Broader Impacts} %
\label{sec:broaderImpacts}


This project is aimed at the Future of Work so the broader impacts are tremendous. Our
interdisciplinary team features personnel from under-represented backgrounds and will be
collaborating with our industry partners to realize significant impacts in society,
education, outreach, and science.
% More goes here


% End of Standard 15 pages, Collaboration Plan is additional 2 pages (if needed)
\newpage
% !TEX root = ../main.tex

% provide a detailed approach for the creation of new knowledge through the rigorous integration of disciplinary knowledge spanning disparate engineering and scientific disciplines. The plan should list the key disciplines needed to achieve the objectives of the proposed research, explain the role of each and why it is necessary, identify the PI or Co-PIs who represent one or more disciplines as experts, outline the approach to the integration and management of these disciplines/experts, and suggest the advances in knowledge that their integration will yield. If a Project Manager is proposed, her or his activities in the project must be described in the Collaboration Plan.

\section*{Collaboration Plan} %
\label{sec:collaborationPlan}

% 2 Pages (16-17)
Our scientific goals are to understand the future of work, through the
lens of software engineering, so we have united expertise from psychology, software
engineering research, and human-computer interaction.
% More goes here

\subsection*{Specific Roles}
The research project will be primarily managed and directed by Dr. ????, at SRI
International, and ???, at ???? College.
% More goes here

\subsection*{Coordination Mechanisms}
The project will utilize both digital and physical coordination mechanisms to enhance ....
% More goes here

\paragraph{Digital Collaboration Mechanisms}

The key personnel have already been actively collaborating and utilizing digital
collaboration tools for discussions and writing.

For instance, we actively use tools for ...
% More goes here

\paragraph{Physical Collaboration Mechanisms}
Physical collaboration and coordination will be handled across numerous conference...
% More goes here

\subsection*{Benefits for Mentoring}
Our proposed research will be conducted across a private liberal arts college and a research institute....
% More goes here

\paragraph{Reference to Budget}
The majority of the coordination mechanisms are either free or currently costed to our institutions (e.g., GitHub, Skype for Business)......
% More goes here
\newpage


% (e) References
% \rfoot{E-\thepage}
\rfoot{E-References}
% \setcounter{page}{1}
\bibliography{main}
\newpage

% (f) Biographical Sketches
\rfoot{F-Biographical Sketch}
% \setcounter{page}{1}
% \input{sections/biosketch_Sanchez.tex}
% \newpage
%!TEX root = ../proposal.tex
\begin{center}
  \textbf{NSF Biographical Sketch} \\
  \textbf{Full name, Ph.D.} \\
  SRI International \textbullet\ email@sri.com \textbullet\ phone
  \end{center}
  
  \subsection*{Professional Preparation}
      SRI International, Menlo Park, CA;
          Computer Science Laboratory;
          Title, From-To
          \\
      University of XYZ, City;
          Computer Science;
          Ph.D., When?
  
  \subsection*{Appointments}
  
  \begin{itemize}%[label={\quad 9999--9999:},leftmargin=*,itemsep=0pt]
  \item 2017-present:
    \textbf{Research Scientist},
    Which Company?,
    Which City and State?
  \end{itemize}
  
  \subsection*{Products}
  \textbf{(i) Five most closely related to the proposed project}
  
  \begin{itemize}%[itemsep=5pt]
  
    \item One
    \item Two
    \item Three
    \item Four
    \item Five \\
  \end{itemize}
  
  \noindent\textbf{(ii) Additional significant publications}
  \begin{itemize}%[itemsep=5pt]
  
    \item One
    \item Two
  \end{itemize}
  
  \subsection*{Synergistic Activities}
  
  \begin{itemize}%[itemsep=4pt]
      \item \textbf{Conference Reviewer:}
      13th International Colloquium on Theoretical Aspects of Computing (ICTAC),
      \item \textbf{Mentoring:}
          (i) Mentor for 1 graduate student, Department of Computer Science, XYZ, City (2014-2015).
          % (ii) Mentor for 1 graduate student, Computer Science Laboratory, SRI International (\DTMdisplaydate{2017}{04}{01}{-1} -- \DTMdisplaydate{2017}{09}{20}{-1}).
  
  \end{itemize}
  

\newpage
% %!TEX root = ../proposal.tex
\begin{center}
  \textbf{NSF Biographical Sketch} \\
  \textbf{Full name, Ph.D.} \\
  SRI International \textbullet\ email@sri.com \textbullet\ phone
  \end{center}
  
  \subsection*{Professional Preparation}
      SRI International, Menlo Park, CA;
          Computer Science Laboratory;
          Title, From-To
          \\
      University of XYZ, City;
          Computer Science;
          Ph.D., When?
  
  \subsection*{Appointments}
  
  \begin{itemize}%[label={\quad 9999--9999:},leftmargin=*,itemsep=0pt]
  \item 2017-present:
    \textbf{Research Scientist},
    Which Company?,
    Which City and State?
  \end{itemize}
  
  \subsection*{Products}
  \textbf{(i) Five most closely related to the proposed project}
  
  \begin{itemize}%[itemsep=5pt]
  
    \item One
    \item Two
    \item Three
    \item Four
    \item Five \\
  \end{itemize}
  
  \noindent\textbf{(ii) Additional significant publications}
  \begin{itemize}%[itemsep=5pt]
  
    \item One
    \item Two
  \end{itemize}
  
  \subsection*{Synergistic Activities}
  
  \begin{itemize}%[itemsep=4pt]
      \item \textbf{Conference Reviewer:}
      13th International Colloquium on Theoretical Aspects of Computing (ICTAC),
      \item \textbf{Mentoring:}
          (i) Mentor for 1 graduate student, Department of Computer Science, XYZ, City (2014-2015).
          % (ii) Mentor for 1 graduate student, Computer Science Laboratory, SRI International (\DTMdisplaydate{2017}{04}{01}{-1} -- \DTMdisplaydate{2017}{09}{20}{-1}).
  
  \end{itemize}
  

% \newpage

% (g) Budget and Budget Justification - Administrative

% (h) Current and Pending Support
% \rfoot{H-\thepage}
% \setcounter{page}{1}
% \input{sections/currentPending}
% \newpage

% (i) Facilities, Equipment, and Other Resources
% \rfoot{I-\thepage}
\rfoot{I-Facilities}
\setcounter{page}{1}
%!TEX root = ../main.tex

\section*{Facilities, Equipment, and Other Resources}
% An aggregated description of the internal and external resources (both physical and personnel) that the organization and its collaborators will provide to the project, should it be funded, has been included.

\subsection*{SRI International}
SRI is an independent, multidisciplinary research facility chartered by the State of
California as an independent, nonprofit research institute. The proposed research
(including development, human subjects testing, and data analyses) will be conducted by
using the facilities of the Computer Science Laboratory (CSL) at SRI.

CSL maintains state-of-the-art computing environments for research and development work....
% More goes here

SRI's intellectual environment is rich with other government funded investigators whose
work may be used complementary to what is proposed in this project (See a partial list
below). These investigators will provide constructive criticism and informal intellectual
input, further amplifying the proposed project's likelihood of success.

\begin{table}[h!]
\centering
\small
\label{lbl:investigators}
\begin{tabular}{lllp{5cm}}
\textbf{Investigator} & \textbf{Agency} & \textbf{Grant Number} & \textbf{Title} \\
Dr. ??? & DARPA           & ????                   & ???           \\
\end{tabular}
\end{table}

SRI has existing technology and infrastructure for performing large scale ...
% More goes here

\subsection*{Name of Partner Institution}
Partner Institution is a selective private liberal arts college in ....

The proposed research (including development and user testing) will be conducted using ...
% More goes here
\\

These facilities, together with those intellectual/collaborative resources, collectively provide a
scientific environment that is strongly supportive of the proposed research and, therefore, success
of the project.
\newpage


% (j) Supplementary Documentation
\rfoot{J-Supplementary}
\setcounter{page}{1}
%!TEX root = ../main.tex

\section*{Data Management Plan}
The majority of the work will be based on open source tools and with publicly available data.
For instance, we will be analyzing user-specific data that is publicly available on GitHub.
There will also be a number of experiments that require ...


% 2 Page Limit
\subsection*{Types of Data and Samples}
% The data we will be acquiring is rather innocuous and the experiments are minimal risk, in that they will likely not involve doing anything beyond what participants might do in their day to day lives.
All data will be connected to the participant's GitHub account.
% More goes here

\paragraph{Policies for Access and Sharing; Protection of Privacy and Confidentiality}
To allow for collaborative data analysis across institutes, electronic data will be hosted on encrypted and password-protected servers at SRI.
The team members at SRI have significant expertise in protection and encryption of sensitive information that must be carefully shared and permission-controlled.
% More goes here

\paragraph{Plans for Storing and Archiving Data}
All electronic data will be moved from the primary computers..
% More goes here

\paragraph{Protection of Privacy and Confidentiality}
Assignment of a random identifier to each participant will preserve privacy and confidentiality.
% More goes here


\newpage

%\includepdf{sections/sjsuSupportLetter.pdf}
%!TEX root = ../main.tex

\section*{Postdoctoral Researcher Mentoring Plan}

The two-year postdoctoral researcher plays a key role in and stands to benefit
significantly from involvement in the proposed work. As our project involves both
technical and social interventions, we can offer a variety of interesting research
questions to interested recent PhDs. Moreover, Dr.....
% More goes here

\subsection*{Career Advising}
TBD

\subsection*{Teaching}
TBD

\subsection*{Training in Research and Mentoring}
TBD

\end{document}